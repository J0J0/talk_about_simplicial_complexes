\chapter{Abstrake simpliziale Komplexe}
\section{Definiton und geometrische Realisierung}
Um nicht an die geometrische Definition von simplizialen Komplexen
im $\R^d$ gebunden zu sein, abstrahieren wir nun von diesem geometrischen Konzept,
um simpliziale Komplexe rein kombinatorisch und unabhägig von euklidschen Räumen
beschreiben zu können.

\begin{thDef}%
    [Abstrakter simplizialer Komplex, abstraktes Simplex, Dimension]\hfill\\
    Ist $E$ eine Menge und $K\subset\pot{E}$, so dass alle Elemente von $K$
    endliche Mächtigkeit haben, so ist $(E,K)$ ein \emph{abstrakter simplizialer
    Komplex}, falls zusätzlich gilt:
    \[ \forall\,\sigma\in K 
        \;\forall\, \sigma'\subset\sigma \colon\; \sigma'\in K
    \]
    Ein Element $\sigma\in K$ nennen wir \emph{(abstraktes) Simplex} und die
    \emph{Dimension von $\sigma$} ist gegeben durch $\dim(\sigma) \defeq \abs{\sigma}$.
    Die \emph{Dimension von $K$} ist das Maximum der Dimensionen seiner Simplizes (oder
    $\infty$, falls dieses nicht existiert):
    \[ \dim(K) \defeq \max_{\sigma\in K}\,\dim(\sigma)  \;\in\N\cup\{\infty\} \]
    Wenn $\abs{E}$ endlich ist, so sprechen wir von einem \emph{endlichen
    simplizialen Komplex}.
\end{thDef}

In Zukunft notieren wir anstatt des Paars $(E,K)$ oft nur noch $K$ und
nehmen an, dass $E=\bigcup K$ gilt. Außerdem können wir aus jedem geometrischen
simplizialen Komplex~$\Delta$ einen abstrakten gewinnen, indem wir wie folgt
vorgehen: Sei $E \defeq E(\Delta)$ die Eckenmenge von $\Delta$ und $K$ wie folgt
gegeben:
\[ K \defeq \bigl\{ E_\sigma 
    \Mid E_\sigma \text{ ist Eckenmenge eines Simplex $\sigma\in\Delta$} \bigr\}
\]
Dann ist $(E,K)$ klarerweise ein abstrakter simplizialer Komplex, welchen wir
eine \emph{geometrische Realisierung von $\Delta$} nennen. Das Polyeder
$\polyeder\Delta$ bezeichnen wir dann auch als ein \emph{Polyeder von $K$}.

% TODO: Beispiel mit Skizze

\begin{thBeispiel}[Graphen als abstrakte simpliziale Komplexe]
    Jeder (einfache, ungerichtete) Graph $(\tilde E,\tilde K)$ mit
    Eckenmenge~$\tilde E$ und Kantenmenge~$\tilde K$ definiert einen
    eindimensionalen simplizialen Komplex~$(E,K)$: Setzte $E\defeq\tilde E$ und
    $K\defeq \{ A \Mid A\subset B\in\tilde K \}$.

    % TODO: Beispielskizze
\end{thBeispiel}

Man sieht relativ einfach ein, dass jeder endliche simpliziale Komplex $(E,K)$ 
eine geometrische Realisierung im $\R^{\abs{E}-1}$ hat, siehe beispielsweise
Matou\v sek\cite[Ch.\,1,\;1.5]{bookc:matousek03}, oder in einer allgemeineren
Variante Munkres\cite[Ch.\,1,\;\S3,\;3.1]{bookc:munkres84}. Stattdessen
verweisen wir auf \cref{asc:geomrealization}, welches eine schärfere Aussage
liefert.


\section{Dimension geometrischer Realisierungen}
Wir wollen nun folgenden Satz beweisen, welcher uns eine obere Schranke für die
nötige Dimension $d\in\N$ gibt, für welche es zu einem abstrakten simplizalen
Komplex eine geometrische Realisierung im $\R^d$ gibt.

\begin{thSatz}%
    [Satz über die maximal nötige Dimension einer geometrischer Realisierung]
    \label{asc:geomrealization}
    %
    Sei $K$ ein abstrakter simplizaler Komplex der Dimension $\dim(K)=d\in\N$.
    Dann besitzt $K$ eine geometrische Realisierung im $\R^{2d+1}$.
\end{thSatz}

Zum Beweis dieses Satzes benötigen wir noch einige Hilfsmittel. 

\begin{thLemma}
    Sei $d\in\N,\; (E,K)$ ein abstrakter simplizialer Komplex und
    $\colon E\to\R^d$ eine injektive Abbildung mit folgender Eigenschaft:
    \[ \forall\,F,G\in K\colon \; f(F\cup G) \text{ ist eine Menge affin
        unabhängiger Vektoren}
    \]
    Dann ist 
    \[ \bigl\{ \conv\blg(f(F)\bigr) \Mid F\in K \bigr\} \]
    eine geometrische Realisierung von $K$ im $\R^d$.
\end{thLemma}

\begin{proof}
    Seien $d,(E,K),f$ wie in der Behauptung und seien $F,G\in K$. 
    Da $f(F\cup G)$ nach Voraussetzung affin unabhängig ist, definiert 
    $\sigma \defeq \conv\bigl(f(F\cup G)\bigr)$ ein Simplex im $\R^d$. 
    Außerdem gilt offenbar $f(F), f(G)\subset f(F\cup G)$, weshalb
    $\sigma_F \defeq \conv\bigl(f(F)\bigr)$ und 
    $\sigma_G \defeq \conv\bigl(f(G)\bigr)$ Seiten von $\sigma$
    sind. Da die Seiten eines Simplex nach \cref{gsc:complexofsimplex} einen
    simplizialen Komplex bilden, ist auch $\sigma_F\cap\sigma_G$ eine Seite von
    $\sigma$, etwa $\conv\bigl(f(T)\bigr)$ für $T\in K$.
\end{proof}<++>
