%
\begin{tikzpicture}[remember picture,overlay]
    \node [yshift=-1.3cm,color=black!40] at (current page.north)
        {%
        \hfill%
        Seminar \enquote{Topologie vs. Kombinatorik} im
        WS~2013/14 an der Universität Regensburg%
        \hfill\mbox{}%
        };
\end{tikzpicture}

\vspace*{-0.5cm}
\begin{center}
    \Large Simpliziale Komplexe
\end{center}

\medskip\noindent
J. Prem (\href{mailto:Johannes.Prem@stud.uni-regensburg.de}%
{\texttt{Johannes.Prem@stud.uni-regensburg.de}})
\hfill
22.~Oktober 2013
\\[-8pt]
\rule{\textwidth}{0.4pt}

\smallskip\noindent
%
Um topologische und kombinatorische Aspekte der Mathematik zusammenzuführen,
benötigen wir als Schnittstelle sogenannte \emph{simpliziale Komplexe}. Wir
werden daher \emph{geometrische} sowie \emph{abstrakte simpliziale Komplexe}
einführen und auf deren Zusammenhang eingehen.
