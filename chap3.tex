\chapter{Quotienten von simplizialen Komplexen}
Wir wollen nun zeigen, dass sich Quotienten von Polyedern nach kontraktiblen
Teilpolyedern (induziert von einem Teilkomplex) in ihrem Homotopietyp nicht vom
Ausgangsraum unterscheiden. Da wir aus aus dem letzten Semester\footnote{%
    Siehe Löh\cite[Kap.\,3,\;3.2,\;3.24\;u.\,3.28]{lecnotes:loeh:at1}, bzw. die
    entsprechende Übungsaufgabe. Alternativ findet man einen Beweis auch bei
    Hatcher\cite[Ch.\,0,\;0.17]{bookc:hatcher02}.%
}
schon allgemeiner wissen, dass dies gilt, falls die Inklusion des Teilraums eine
Kofaserung ist, zeigen wir also folgende Aussage, woraus wir direkt das
anschließende gewünschte Korollar erhalten:

\begin{thLemma}[Inklusion eines Teilkomplexes ist Kofaserung]
    \label{quot:inclusioncofib}
    %
    Sei $K$ ein endlicher simplizialer Komplex und $L\subset K$ ein
    Teilkomplex.  Dann ist die Inklusion zugehöriger Polyeder (einer
    geometrischen Realisierung) $\polyeder L \hookrightarrow \polyeder K$ eine
    Kofaserung.
\end{thLemma}

\begin{thKorollar}%
    [Quotienten von simplizialen Komplexen nach kontraktiblen Teilkomplexen]
    %
    Sei $K$ ein endlicher simplizialer Komplex und $L\subset K$ ein
    Teilkomplex, für den $\polyeder L$ kontraktibel ist. Dann sind $\polyeder K$
    und $\polyeder K\mkern2mu/\mkern2mu\polyeder L$ homotopieäquivalente Räume.
\end{thKorollar}

Bevor wir \cref{quot:inclusioncofib} zeigen können, brauchen wir zuerst noch
ein Hilfsresultat, das aber auch für sich allein genommen interessant ist:

\begin{thLemma}[Abgeschlossenheit von Teilkomplexen]
    \label{quot:subcomplexclosed}
    %
    Sei $\Delta$ ein (geometrischer) simplizialer Komplex und sei
    $\Delta'\subset\Delta$ ein Teilkomplex. Dann ist $\polyeder{\Delta'}$
    abgeschlossen in $\polyeder\Delta$.
\end{thLemma}

\begin{proof}
    Sei $\sigma\in\Delta$. Dann gilt:
    \[ \polyeder{\Delta'}\cap\sigma 
        = \bigcup\, \bigl\{ \sigma'\in\Delta(\sigma) \cMid\big \sigma'\in\Delta'
        \bigr\}
    , \]
    d.\,h. $\polyeder{\Delta'}\cap\sigma$ ist die Vereinigung aller Seiten von
    $\sigma$, welche in $\Delta'$ enthalten sind. Jedes der obigen $\sigma'$ ist
    aber abgeschlossen in $\sigma$ (als kompakte Teilmenge eines Hausdorffraums)
    und damit ist die endliche Vereinigung abgeschlossener Mengen auf der rechten
    Seite ebenfalls abgeschlossen in $\sigma$. Da $\sigma$ beliebig war, ist
    $\polyeder{\Delta'}$ also abgeschlossen in $\Delta$ nach Definition der
    Topologie auf $\polyeder\Delta$.
    \\
\end{proof}

%\begin{thLemma}[Stetigkeit einer Abbildung aus einem Polyeder heraus]
%    Ist $\Delta$ ein (geometrischer) simplizialer Komplex, $X$ ein topologischer
%    Raum und $f\colon\polyeder\Delta\to X$ eine Abbildung, so gilt:
%    \[ f\text{ stetig} 
%        \qiffq \forall\,\sigma\in\Delta\colon\; f\vert_\sigma \text{ stetig}
%    \]
%\end{thLemma}
%
%Der Beweis ist eine einfache Übung und ergibt sich direkt aus der Topologie auf
%$\polyeder\Delta$. Wir widmen uns nun dem Beweis von \cref{quot:inclusioncofib}.

\medskip
\begin{proof}[Beweis von \cref{quot:inclusioncofib}]
    \belowpdfbookmark{Beweis von Lemma \ref{quot:inclusioncofib}}{proofofinclusioncofib}
    %
    Sei $K$ ein endlicher simplizialer Komplex und $L\subset K$ ein
    Teilkomplex. Um die Notation übersichtlicher zu halten, bezeichnen wir die
    Polyeder von $K$ bzw. $L$ mit $X$ bzw. $A$. Weiter bezeichne im Folgenden
    $I\defeq[0,1]$ das Einheitsintervall.

    Da $A$ nach \cref{quot:subcomplexclosed} abgeschlossen in $X$ ist, genügt es
    zu zeigen, dass $(A\times I)\cup(X\times\{0\})$ ein Retrakt von $X\times I$
    ist, d.\,h. dass es eine stetige Abbildung
    \begin{gather*}
        r\colon X\times I \to (A\times I)\cup(X\times\{0\}) 
        \qquad\text{mit} 
        \\ 
        r \circ \bigl( (A\times I)\cup(X\times\{0\}) \hookrightarrow
        X\times I \bigr) = \id_{(A\times I)\cup(X\times\{0\})}
    \end{gather*}
    gibt.\footnote{%
        Man kann zeigen, dass die Bedingung, dass $A$ abgeschlossen ist, nicht
        nötig ist, dass wir also auf \cref{quot:subcomplexclosed} hätten
        verzichten können. Der Beweis dafür ist aber recht kompliziert, wie man
        beispielsweise bei Hatchter\cite[Appendix,\,A.18]{bookc:hatcher02}
        nachlesen kann.%
    }
    Dazu gehen wir wie folgt vor:

    \pagebreak[2]
    Zuerst betrachten wir den Fall, dass $X$ ein Simplex und $A$ der Rand
    des Simplex ist. Sei $n\defeq\dim(X)$. Für $n=-1$ gibt es nichts zu tun und
    für $n=0$ ist $A=\emptyset$, also können wir die Projektion (auf
    $X\times\{0\}$) als Retraktion nehmen. Sei also nun $n\geq1$.
    Dann wissen wir nach \cref{gsc:convexsethomeodisk}, dass es
    einen Homöomorphismus 
    $f\colon X\xrightarrow{\smash{\raisebox{-2pt}{$\sim$}}} D^n$ 
    gibt, welcher $A$ homöomorph auf $S^{n-1}\subset D^n$ abbildet.
    Wir betrachten dann folgendes Diagramm:
    \begin{equation*}
        \begin{xy}
            \xymatrix@C=1.5cm{
                X\times I \ar@{-->}[r] \ar[d]^{f\times\id_I}
                & (A\times I)\cup(X\times\{0\})    
                \\
                D^n\times I \ar[r] 
                & (S^{n-1}\times I)\cup(D^n\times\{0\}) \ar[u]^g
            }
        \end{xy}
    \end{equation*}
    Die untere horizontale Abbildung sei eine Retraktion von $D^n\times I$ auf
    $(S^{n-1}\times I)\cup(D^n\times\{0\})$, welche existiert, da
    $S^{n-1}\hookrightarrow D^n$ eine Kofaserung ist.\footnote{%
        Oder man nehme eine explizite Retraktion zum Beispiel von
        tom Dieck\cite[Ch.\,2,\;2.3.5]{bookc:tomdieck08}, welche bei diesem
        gerade zum Beweis dieser letzten Aussage benutzt wird.%
    }
    Für $g$ nehmen wir die
    verklebte Abbildung 
    \[ g \defeq \bigl(f\vert_A^{-1} \times\id_I\bigr) \cup_{S^{n-1}\times\{0\}}
        \bigl( f^{-1}\times\const_0 \bigr)
    , \]
    wobei $S^{n-1}\times\{0\}$ abgeschlossen in $D^n\times\{0\}$ und somit $g$
    stetig ist. Wir erhalten also die gestrichelte Abbildung, die gesuchte
    Retraktion, indem wir das Diagramm unten herum durchlaufen.

    Nun zum allgemeinen Fall. Ohne Einschränkung gelte 
    $\emptyset\neq L \subsetneq K$. Ist $\Delta$ die geometrische Realisierung
    von $K$ zu welchem $X$ das Polyeder ist, dann gilt:
    \[ X \times I 
        = \Bigl(\, \bigcup_{\sigma\in\Delta} \sigma \mkern2mu\Bigr) \times I
        = \bigcup_{\sigma\in\Delta} (\sigma\times I)
    \]
    Betrachte dann ein Simplex maximaler Dimension aus $K\setminus L$, bzw. das
    zugehörige geometrische Simplex~$\sigma$. Sei $r'$ dann eine Retraktion wie
    aus dem vorherigen Abschnitt, welche wir mit der Identität auf
    $\bigcup_{\sigma'\in\Delta\setminus\{\sigma\}} (\sigma'\times I)$ verkleben
    können. Da es nur endlich viele solcher Simplizes~$\sigma$ gibt, können wir
    dies der Reihe nach mit allen derartigen Simplizes durchführen bis es keine
    Simplizes aus $K\setminus L$ mehr gibt, welche noch \enquote{aufgedickt}
    sind. Dieser Prozess liefert letztlich also eine Retraktion~$r$ wie
    gewünscht.
    \\
\end{proof}
