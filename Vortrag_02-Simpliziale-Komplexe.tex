\documentclass[11pt,a4paper,ngerman,DIV=11,bibliography=totoc]{scrreprt}

%%%%%%%%%%%%%%%%%%%%%%%%%%%%%%%%%%%%%%%%%%%%%%%%%%%%%%%%%%%%%%%%%%%%%
%%% packages
%%%%%%%%%%%%%%%%%%%%%%%%%%%%%%%%%%%%%%%%%%%%%%%%%%%%%%%%%%%%%%%%%%%%%

\usepackage[utf8]{inputenc}
\usepackage[T1]{fontenc}
\usepackage[ngerman]{babel}

\usepackage{amsmath}
\usepackage{amssymb}
\usepackage{amsthm}
\usepackage{mathtools}
\usepackage[all]{xy}

\usepackage[babel]{csquotes}
\usepackage[shortlabels]{enumitem}
\usepackage[numbers,sort&compress]{natbib}
\usepackage{ifmtarg}
\usepackage{xstring}
\usepackage{remreset}


\usepackage[pdftex,bookmarks,colorlinks=false,pdfborder={0 0 0},%
            pdftitle={Seminar Topologie vs. Kombinatorik - %
                      Vortrag 2: Simpliziale Komplexe},%
            pdfauthor={Johannes Prem}]{hyperref}
%
\usepackage{cleveref}
\let\cref=\Cref

\usepackage{helpers} % my own helpers.sty


%%%%%%%%%%%%%%%%%%%%%%%%%%%%%%%%%%%%%%%%%%%%%%%%%%%%%%%%%%%%%%%%%%%%%
%%% macro definitions and other things
%%%%%%%%%%%%%%%%%%%%%%%%%%%%%%%%%%%%%%%%%%%%%%%%%%%%%%%%%%%%%%%%%%%%%

% don't reset footnote numbers
\makeatletter
\@removefromreset{footnote}{chapter}
\makeatother


% make parenthesized versions of \ref and cleveref's \cref
\newcommand*{\pref}[1]{(\ref{#1})}
\newcommand*{\pcref}[1]{(\cref{#1})}

% make a even more clever \mycref that produces "Lemma 42a)" etc.
% (to see it in action check out the code in chap{1,2,3}.tex)  % TODO
\newcommand{\mycref}[1]{%
    \begingroup%
    \StrCount{#1}{:}[\mycrefCount]%
    \StrBefore[\mycrefCount]{#1}{:}[\myrefMain]%
    \expandafter\cref\expandafter{\myrefMain}\,\ref{#1}%
    \endgroup%
}

% make \varepsilon and \varphi default
\varifygreekletters{\epsilon\phi}

% change the qedsymbol to my favoured blacksquare
\renewcommand{\qedsymbol}{$\blacksquare$}

% style for /all/ theorem like environments
\newtheoremstyle{mythms}
 {15pt}% space above
 {12pt}% space below 
 {}% body font
 {}% indent amount
 {\bfseries}% theorem head font
 {.}% punctuation after theorem head
 {0.6cm plus 0.25cm minus 0.1cm}% space after theorem head (\newline possible)
 {}% theorem head spec 
 
% set style and define thm like environments
\theoremstyle{mythms}
\newtheorem{globalnum}{DUMMY DUMMY DUMMY}[chapter]
\newtheorem{thDef}[globalnum]{Definition}
\newtheorem{thNotation}[globalnum]{Notation}
\newtheorem{thSatz}[globalnum]{Satz}
%\newtheorem{thPropos}[globalnum]{Proposition}
\newtheorem{thLemma}[globalnum]{Lemma}
\newtheorem{thKorollar}[globalnum]{Korollar}

\newtheorem{thBemerkung}[globalnum]{Bemerkung}
%\newtheorem{thWarnung}[globalnum]{Warnung}
\newtheorem{thBeispiel}[globalnum]{Beispiel}
\newtheorem{thBeispiele}[globalnum]{Beispiele}
\newenvironment{BspList}[1][]{%
\nopagebreak\begin{thBeispiele}#1%
\hfill\begin{enumerate}[a),parsep=0pt,itemsep=0.8ex,leftmargin=2em]%
}{%
\end{enumerate}\end{thBeispiele}
}
%

% also define a 'proofsketch' version of 'proof'
\newenvironment{proofsketch}[1][]{%
\begin{proof}[Beweisskizze#1]
}{%
\end{proof}
}

% inject pdfbookmarks at thm like environments
\makeatletter
\let\origthmhead=\thmhead
\renewcommand{\thmhead}[3]{%
\origthmhead{#1}{#2}{#3}%
\belowpdfbookmark{#1\@ifnotempty{#1}{ }#2\thmnote{ (#3)}}{#1#2}%
}
\makeatother

% new math operators
\DeclareMathOperator*{\bigdotcup}{\overset{\mkern0mu\scalebox{0.6}{$\bullet$}}{\bigcup}}

% new math 'operators'  % TODO update
\DeclareMathOperator{\const}{const}
\DeclareMathOperator{\conv}{conv}
\DeclareMathOperator{\id}{id}
\DeclareMathOperator{\Image}{im}
\DeclareMathOperator{\Kernel}{ker}
\DeclareMathOperator{\powerset}{\mathcal{P}}
\DeclareMathOperator{\supp}{supp}

% make quantors that use \limits per default
\DeclareMathOperator*{\Exists}{\exists}
\DeclareMathOperator*{\forAll}{\forall}

% define an 'abs', 'norm' and 'Spann' command
\DeclarePairedDelimiter{\abs}{\lvert}{\rvert}
\DeclarePairedDelimiter{\norm}{\lVert}{\rVert}
\DeclarePairedDelimiter{\Spann}{\langle}{\rangle}

\let\polyeder=\norm

%---
% the following creates an operator norm as a tripple stroke \vert
% (source: mathabx fonts)
\DeclareFontFamily{U}{matha}{\hyphenchar\font45}
\DeclareFontShape{U}{matha}{m}{n}{
      <5> <6> <7> <8> <9> <10> gen * matha
      <10.95> matha10 <12> <14.4> <17.28> <20.74> <24.88> matha12
      }{}
\DeclareSymbolFont{matha}{U}{matha}{m}{n}
\DeclareFontFamily{U}{mathx}{\hyphenchar\font45}
\DeclareFontShape{U}{mathx}{m}{n}{
      <5> <6> <7> <8> <9> <10>
      <10.95> <12> <14.4> <17.28> <20.74> <24.88>
      mathx10
      }{}
\DeclareSymbolFont{mathx}{U}{mathx}{m}{n}

\DeclareMathDelimiter{\vvvert}{0}{matha}{"7E}{mathx}{"17}
\DeclarePairedDelimiter{\opnorm}{\vvvert}{\vvvert}
%---

% define missing arrows
\newcommand{\longto}{\longrightarrow}
\newcommand{\longhookrightarrow}{\lhook\joinrel\relbar\joinrel\rightarrow}

% provide mathbb symbols \N \Z \Q \R and \C
\defmathbbsymbols{Z Q C}
\defmathbbsymbolsubs{N R}

% define some set specific macros
\newcommand{\setclosure}[1]{\overline{#1}}
\newcommand{\setinterior}[1]{#1^\circ}
\newcommand{\setboundary}[1]{\partial #1}

% just some shortcuts
\newcommand{\defeq}{\coloneqq}
\newcommand{\eqdef}{\eqqcolon}
\newcommand{\half}{\frac{1}{2}}
\newcommand{\isum}[1][0]{\sum_{i=#1}}
\newcommand{\mr}{\mathrm}
%\newcommand{\mt}{^\mathsf{t}}
%\newcommand{\Nfolge}[1]{\left(#1_n\right)_{n\in\N}}
\newcommand{\pot}[1]{\powerset(#1)}
\newcommand{\qiffq}{\quad\iff\quad}
\newcommand{\qimpliesq}{\quad\implies\quad}
\newcommand{\qoderq}{\qtextq{oder}}
\newcommand{\qqtextqq}[1]{\qquad\text{#1}\qquad}
\newcommand{\qqundqq}{\qqtextqq{und}}
\newcommand{\qtextq}[1]{\quad\text{#1}\quad}
\newcommand{\qundq}{\qtextq{und}}
\newcommand{\setOneto}[1]{\{1,\ldots,#1\}}
\newcommand{\setZeroto}[1]{\{0,\ldots,#1\}}
\newcommand{\thalf}{\tfrac{1}{2}}

%
\newcommand{\Achtung}{\emph{Achtung:} }
\newcommand{\skeleton}[2]{#1^{\leq#2}}

% overwrite \Re and \Im with less fancier definitions
\DeclareMathOperator{\Realteil}{Re}
\DeclareMathOperator{\Imaginaerteil}{Im}
\let\Re=\Realteil
\let\Im=\Imaginaerteil


% make a \Mid macro as flexible replacement for \mid in set definitions
\newcommand{\Mid}[1][\,]{%
#1%
\ifnum\currentgrouptype=16%
\middle\vert\else\vert\fi%
#1%
}

% \dots and a version for custom size control
\newcommand{\cMid}[2][\,]{%
#1#2\vert#1%
}


% xy tip selection (ComputerModern)
\SelectTips{cm}{}
\UseTips

% listing with -- is nicer than with bullets 
\setlist[itemize,1]{label=--}

% start at chapter 0
\setcounter{chapter}{-1}

%%%%%%%%%%%%%%%%%%%%%%%%%%%%%%%%%%%%%%%%%%%%%%%%%%%%%%%%%%%%%%%%%%%%%
%%% document
%%%%%%%%%%%%%%%%%%%%%%%%%%%%%%%%%%%%%%%%%%%%%%%%%%%%%%%%%%%%%%%%%%%%%

\begin{document}


\subject{Seminar: Topologie vs. Kombinatorik}
\title{Simpliziale Komplexe}
\author{Johannes Prem}
\date{??.??.2013}

\maketitle


%\chapter{Vorwort, Notation und Konventionen}
\chapter{Vorwort und Notation}
Um topologische und kombinatorische Aspekte der Mathematik zusammenzuführen,
benötigen wir eine passende Schnittstelle, die es uns erlaubt, topologische
Räume mit kombinatorischen Mitteln zu untersuchen und umgekehrt aus
kombinatorischen Strukturen topologische Räume zu bauen. Diese Schnittstelle
stellen die sogenannten \emph{simplizialen Komplexe} dar, welche wir im
Folgenden einführen werden. Zunächst in Form geometrischer simplizialer
Komplexe, von welchen wir geeignet abstrahieren werden, um simpliziale Komplexe
in purer Mengentheorie formulieren zu können. Wir werden außerdem betrachten,
wie beide Konzepte zusammenhängen und was simpliziale Komplexe mit partiell
geordneten Mengen zu tun haben. Zuletzt betrachten wir Quotienten simplizialer
Komplexe nach kontraktiblen Teilräumen.


\bigskip
In diesem Skript wird folgende Notation verwendet:
\begin{itemize}
    \item
        Sowohl $\subset$ als auch $\subseteq$ stehen für: enthalten oder gleich.
        Echt enthalten wird durch $\subsetneq$ gekennzeichnet.
    
    \item
        Die \emph{Natürlichen Zahlen $\N$} beginnen mit $0$.
        
    \item
        Zu einer Menge $X$ bezeichnet $\pot X$ die Potenzmenge von $X$.

    \item % TODO: v  bessere Formulierung
        Der $n$-dimensionale Einheitsball wird mit $D^n$ bezeichnet und
        sein Rand $\setboundary{D^n}$, 
        die $(n{-}1)$-dimensionale Einheitssphäre, mit $S^{n-1}$.
        
    \item
        Im Kontext eines $\R^n$ (für ein $n\in\N$) bezeichnet $e_i$ den $i$-ten
        Einheitsvektor.
\end{itemize}











\chapter{Geometrische simpliziale Komplexe}
\section{Definition und erste Eigenschaften} % TODO: besserer Name?
Oftmals ist es hilfreich, einen topologischen Raum in kleinere Bausteine zu
unterteilen. Eine typische solche Unterteilungsmöglichkeit sind \emph{CW-Komplexe}, welche
aber für kombinatorische Anwendungen noch zu allgemein sind.\footnote{%
    In der Tat kann man zeigen, dass es CW-Komplexe gibt, welche sich nicht
    durch die nachfolgenden Konstruktionen beschreiben lassen. Dass dies sogar
    schon mit einem zweidimensionalen CW-Komplex möglich ist, zeigt
    beispielsweise Bognár\cite{artcle:bognar77}. Ein Beispiel für einen
    dreidimensionalen solchen CW-Komplex findet man bei
    Lundell\cite[Ch.\,III,\;1.8]{bookc:lundell69}.%
}
Wir interessieren uns daher dafür, wie wir topologische Räume beschreiben
können, die aus kombinatorisch einfacher zu beschreibenden Bausteinen aufgebaut
sind. Namentlich werden dies \emph{Strecken, Dreiecke, Tetraeder} und
höherdimensionale Objekte desselben Typus sein, welche mathematisch durch
sogenannte \emph{Simplizes} beschrieben werden können. Dazu benötigen wir
zunächst die folgenden Begriffe:

\begin{thDef}[Affin (un-)abhängig]
    Sind $n,d\in\N$ und $v_0,\ldots,v_n\in\R^d$, so bezeichnen wir diese
    Vektoren als \emph{affin abhängig}, falls es Elemente
    $\lambda_0,\ldots,\lambda_n \in\R$ mit folgenden drei Eigenschaften gibt:
    \[ \{\lambda_0,\ldots,\lambda_n\} \neq \{0\}, \quad \isum^n \lambda_i = 0
    \qundq \isum^n \lambda_i\,v_i = 0 \in\R^d.\] 
    Falls es solche Zahlen nicht gibt, so bezeichnen wir die Vektoren als
    \emph{affin unabhängig}.
\end{thDef}

\begin{thLemma}[Äquivalente Bedingung für affine Unabhängigkeit]
    \label{gsc:iffaffinlyindependet}
    %
    Seien $n,d\in\N$. Dann sind $v_0,\ldots,v_n\in\R^d$ genau dann affin
    unabhängig, wenn für ein beliebiges $k\in\setZeroto{n}$ die Familie
    \[  (v_i - v_k)_{i \in\setZeroto{n}\setminus\{k\}} \] 
    von Vektoren im $\R^d$ linear unabhängig ist.
\end{thLemma}

Der Beweis ist eine einfache Übung.
Aus \cref{gsc:iffaffinlyindependet} folgt insbesondere, dass für festes $d\in\N$
maximal $d+1$ Vektoren im $\R^d$ affin unabhängig sein können.

\begin{thDef}[Konvexe Menge, konvexe Hülle, Konvexkombination]
    Sei $d\in\N$. Eine Teilmenge $A\subset\R^d$ heißt \emph{konvex}, wenn
    für alle $x,y\in A$ auch die (lineare) Verbindungsstrecke
    \[ \{ tx + (1-t)\,y \Mid t\in[0,1] \} \]
    in~$A$ enthalten ist.

    \newpage
    \noindent
    Für eine beliebige Teilmenge $A'\subset\R^d$ definiert
    \[ \conv(A') \defeq 
        \bigcap \,\bigl\{ A\subset\R^d \Mid A'\subset A\text{ konvex} \bigr\}
    \]
    die \emph{konvexe Hülle von $A'$}.
    
    \noindent
    Sind außerdem $n\in\N$ und $v_0,\ldots,v_n\in\R^d$ beliebig sowie
    $\lambda_0,\ldots,\lambda_n\in\R[\geq0]$ mit $\isum^n\lambda_i = 1$, so
    bezeichnet man $\isum^n \lambda_i\,v_i$ als \emph{Konvexkombination der
    $v_i$}.
\end{thDef}

\begin{thBemerkung} \label{gsc:convexhullviaconvexcombinations}
    Mithilfe einfacher Argumente zeigt man außerdem, dass für eine Teilmenge
    $A'\subset\R^d$ die konvexe Hülle $\conv(A')$ gerade aus allen 
    Konvexkombinationen je endlich vieler Elemente aus $A'$ besteht.
\end{thBemerkung}


\begin{thDef}[(Geometrisches) Simplex, Eckpunkte, Dimension]
    \label{gsc:def:simplex}
    %
    Ist $d\in\N$ und $A\subset\R^d$ eine endliche Menge affin unabhängiger
    Vektoren im $\R^d$ (oder leer), so bezeichnet man $\sigma \defeq \conv(A)$ als
    \emph{(geometrisches) Simplex} und nennt die Elemente aus $A$
    \emph{Eckpunkte (von~$\sigma$)}, oder kurz \emph{Ecken (von~$\sigma$)}. Die
    \emph{Dimension von $\sigma$} ist dann durch $\dim(\sigma) \defeq \abs{A}-1$
    definiert und man nennt ein $n$-dimensionales Simplex auch kurz
    \emph{$n$-Simplex} (wobei $n\in\N\cup\{-1\}$).
\end{thDef}

\begin{thDef}[Seite eines Simplex]
    Ist $d\in\N$ und $\sigma$ ein Simplex mit der Eckenmenge
    $E \subset\R^d$, so ist die konvexe Hülle $\conv(E')$ einer beliebigen 
    Teilmenge $E'\subset E$ wieder ein Simplex und jedes dieser Simplizes wird
    als \emph{Seite von $\sigma$} bezeichnet.
\end{thDef}

\begin{figure}
    \centering
    \begin{tikzpicture}[%
            scale=1.5,
            mypoint/.style={shape=circle, inner sep=1.8pt, color=black, fill}
        ]
        \begin{scope}
            \node at (0,0) [mypoint] {};
        \end{scope}
        \begin{scope}[shift={(1.5,0)}]
            \draw (0,0) node [mypoint] {} -- (1,0) node [mypoint] {};
        \end{scope}
        \begin{scope}[shift={(4,0)}]
            \draw [fill=black!20] (0,0) node [mypoint] {} -- 
                  (1,0) node [mypoint] {} --
                  (0.5,1) node [mypoint] {} -- cycle ;
        \end{scope}
        \begin{scope}[shift={(6.7,0)},z={(-5mm,-4mm)}]
            \coordinate (A) at (0,0,1);
            \coordinate (B) at (1,0,0);
            \coordinate (C) at (0,1,0);
            \coordinate (D) at (0,0,0);

            \draw [fill=black!20]
                  (A) node [mypoint] {} --
                  (B) node [mypoint] {} --
                  (C) node [mypoint] {} -- cycle
                  (D) node [mypoint] {};
            \draw [dashed] (A) -- (D) -- (C) (D) -- (B);
        \end{scope}
    \end{tikzpicture}
    \caption{$0$- bis $3$-dimensionale Simplizes}
    \label{gsc:fig:simplices}
\end{figure}

\begin{thDef}%
    [Geometrischer simplizialer Komplex, Dimension, Polyeder, Eckenmenge]
    \label{gsc:def:gsc}
    %
    Sei $d\in\N$. Ist $\Delta$ eine nicht-leere Menge von (geometrischen)
    Simplizes im $\R^d$, so ist $\Delta$ ein \emph{(geometrischer) simplizialer
    Komplex}, falls folgende Bedingungen erfüllt sind:
    \begin{itemize}
        \item Ist $\sigma$ ein Simplex in $\Delta$, so ist auch jede Seite von
            $\sigma$ in $\Delta$ enthalten.
        \item
            Der Schnitt zweier Simplizes $\sigma_1,\sigma_2\in\Delta$ ist eine
            Seite von $\sigma_1$ und $\sigma_2$.
    \end{itemize}
    Wir definieren die \emph{Dimension von $\Delta$} als die maximale Dimension
    seiner Simplizies:
    \[ \dim(\Delta) \defeq \max\{ \dim(\sigma) \Mid \sigma\in\Delta \} \;\in\N
    \]
    Ist $\Delta$ ein simplizialer Komplex, so setzen wir
    \[ \polyeder\Delta \defeq \bigcup \Delta \;\subset\R^d \]
    und statten diesen Raum mit der von den Inklusionen der einzelnen Simplizes
    induzierten Finaltopologie aus. Explizit bedeutet das hier:
    $A\subset\polyeder\Delta$ ist genau dann offen (abgeschlossen), wenn für alle
    $\sigma\in\Delta$ der Schnitt $A\cap\sigma$ offen (abgeschlossen) in
    $\sigma$ ist (wobei $\sigma$ die Teilraumtopologie trägt). Wir nennen dann
    $\polyeder\Delta$ das \emph{zu $\Delta$ gehörige Polyeder}.
    Außerdem definieren wir $V(\Delta)$ als die Vereinigung
    aller Eckenmengen von Simplizes in $\Delta$.
\end{thDef}

\Achtung
In der Literatur wird des Öfteren der Fall des leeren Simplex~$\emptyset$
ausgeschlossen. Nach den obigen Definitionen \ref{gsc:def:simplex} und 
\ref{gsc:def:gsc} ist dies aber erlaubt und insbesondere enthält jeder
simpliziale Komplex das leere Simplex.

\begin{figure}
    \centering
    \begin{tikzpicture}[%
            mypoint/.style={shape=circle, inner sep=1.8pt, color=black, fill}
        ]
        \begin{scope}[z={(-5mm,-4mm)}]
            \coordinate (A) at (0,0,1);
            \coordinate (B) at (1,0,0);
            \coordinate (C) at (0,1,0);
            \coordinate (D) at (0,0,0);
            
            \coordinate (E) at (2,0,0);
            \coordinate (F1) at (3,1,0);
            \coordinate (F2) at (3,-1,0);
            \coordinate (G) at (4,0,0);
            \coordinate (H) at (4.5,0.7,0);
            
            \draw [fill=black!20]
                  (A) node [mypoint] {} --
                  (B) node [mypoint] {} --
                  (C) node [mypoint] {} -- cycle
                  (D) node [mypoint] {};
            \draw [dashed] (A) -- (D) -- (C) (D) -- (B);
            
            \draw (B) -- (E);
            \draw [fill=black!20] (E) -- (F1) -- (F2) -- cycle;
            \draw [fill=black!20] (G) -- (F1) -- (F2) -- cycle;
            \path (E)  node [mypoint] {} 
                  (F1) node [mypoint] {}
                  (F2) node [mypoint] {};
            
            \draw (G) node [mypoint] {} -- (H) node [mypoint] {};
        \end{scope}
        
        \begin{scope}[shift={(7,1)},scale=0.6]
            \draw (-1,0) node [mypoint] {} -- (1,0) node [mypoint] {}
                  (0,-1) node [mypoint] {} -- (0,1) node [mypoint] {};
        \end{scope}
        
        \begin{scope}[shift={(6,0)}]
            \coordinate (E) at (2,0,0);
            \coordinate (F1) at (3,1,0);
            \coordinate (F2) at (3,-1,0);
            \coordinate (F3) at (3,0,0);
            \coordinate (G) at (4,0,0);
            
            \draw [fill=black!20] (E) -- (F1) -- (F2) -- cycle;
            \draw [fill=black!20] (G) -- (F1) -- (F3) -- cycle;
            \path (E)  node [mypoint] {} 
                  (F1) node [mypoint] {}
                  (F2) node [mypoint] {}
                  (G)  node [mypoint] {};
        \end{scope}
        
        \begin{scope}[shift={(10.5,-1)}]
            \draw [fill=black!20] 
                (0,0) node [mypoint] {} --
                (2,0) node [mypoint] {} -- 
                (1,1) node [mypoint] {};
            \draw (1,0) node [mypoint] {} -- (1.5,-0.5) node [mypoint] {};
        \end{scope}
    \end{tikzpicture}
    \caption{Simplizialer Komplex (links) und 
             \emph{keine} simplizialen Komplexe (rechts)}
    \label{gsc:fig:simplicalcomplex}
\end{figure}

\begin{thDef}[Rand und Inneres eines Simplex]
    Ist $\sigma$ ein Simplex, so bezeichnen wir die Vereinigung aller Seiten von
    $\sigma$ mit Dimension echt kleiner als $\dim(\sigma)$ als \emph{Rand von
    $\sigma$} und die Teilmenge von $\sigma$, die sich ergibt, wenn wir aus
    $\sigma$ den Rand entfernen, als \emph{(relatives) Inneres von $\sigma$}.
\end{thDef}

\begin{thDef}[Träger]
    Ist $\Delta$ ein simplizialer Komplex und $x\in\polyeder\Delta$, so gibt es
    genau ein Simplex $\sigma\in\Delta$, für das $x$ im Inneren von $\sigma$
    liegt. Dieses bezeichnen wir als \emph{Träger von~$x$} und schreiben dafür
    $\supp(x)$.
\end{thDef}

\begin{thDef}[Teilkomplex, \texorpdfstring{$k$}{k}-Skelett]
    Sei $\Delta$ ein simplizialer Komplex. Wir nennen $\Delta'\subset\Delta$
    einen \emph{Teilkomplex}, falls $\Delta'$ selbst einen simplizialen Komplex
    definiert.
    Für jedes $k\in\setZeroto{\dim(\Delta)}$ ist
    \[ \skeleton\Delta{k} 
        \defeq \bigl\{ \sigma\in\Delta \Mid \dim(\sigma)\leq k \bigr\}
    \]
    ein Teilkomplex von $\Delta$, welchen wir \emph{$k$-Skelett von $\Delta$}
    nennen. (Insbesondere gilt für $k=0$: $\skeleton\Delta0 = V(\Delta)$.)
\end{thDef}

\bigskip
Im Folgenden werden wir hauptsächlich \emph{endliche} simpliziale Komplexe
betrachten, d.\,h. simpliziale Koplexe mit nur endlich vielen Simplizes. Daher
halten wir noch folgende Aussage fest:

\begin{thLemma}[Polyeder eines endlichen simplizialen Komplexes]
    Ist $d\in\N$ und $\Delta$ ein endlicher simplizialer Komplex im $\R^d$, 
    so stimmt die Topologie auf $\polyeder\Delta$ aus \cref{gsc:def:gsc} mit 
    der Teilraumtopologie von $\polyeder\Delta$ als Teilmenge des $\R^d$
    überein.
\end{thLemma}

Der Beweis ist einfach und wird an dieser Stelle dem Leser zur Übung überlassen.
Einfache aber nützliche Beispiele für endliche simpliziale Komplexe liefert das
folgende Lemma:

\begin{thLemma}[Simplizialer Komplex eines Simplex]
    \label{gsc:complexofsimplex}
    %
    Sind $n,d\in\N$ und ist $\sigma$ ein $n$-Simplex im $\R^d$, so ist
    \[ \Delta(\sigma) 
        \defeq \{ \sigma' \Mid \sigma' \text{ ist Seite von } \sigma \}
    \]
    ein $n$-dimensionaler (endlicher) simplizialer Komplex im $\R^d$.
\end{thLemma}

\begin{proofsketch}
    Falls $\Delta(\sigma)$ ein simplizialer Komplex ist, ist klar, dass dieser
    endlich ist, da die Anzahl der Ecken von $\sigma$ endlich ist. Außerdem gilt
    $\sigma\in\Delta(\sigma)$, womit $\dim\bigl(\Delta(\sigma)\bigr) = n$ klar
    ist. Es bleibt also zu zeigen, dass $\Delta(\sigma)$ einen simplizialen
    Komplex definiert. Die erste Bedingung in \cref{gsc:def:gsc} ist dabei
    offensichtlich erfüllt und für die zweite zeigt man mithilfe von
    \cref{gsc:convexhullviaconvexcombinations} und einfachen Umformungen, dass
    für Teilmengen $A,B$ der Eckenmenge von $\sigma$ stets $\conv(A)\cap\conv(B)
    = \conv(A\cap B)$ gilt.
    \\
\end{proofsketch}

Das letzte Argument findet man ausführlicher auch bei 
Matou\v sek\cite[Ch.\,1,\;1.3.6]{bookc:matousek03}.


\section{Triangulierung}
\begin{thDef}[Triangulierbar, Triangulierung]
    Ein beliebiger topologischer Raum~$X$ heißt \emph{triangulierbar}, falls es
    einen simplizialen Komplex $\Delta$ und einen Homöomorphismus
    $\polyeder\Delta \cong X$ gibt. Solch einen Homöomorphismus nennen wir 
    eine \emph{Triangulierung von $X$}.
\end{thDef}

Die folgenden Aussagen liefern uns Triangulierungen für Sphären jeder
Dimension:

\begin{thLemma}[Kompaktheit konvexer Hülle endlich vieler Punkte]
    \label{gsc:convexhullcompact}
    %
    Ist $d\in\N$ und $A\subset\R^d$ eine endliche Teilmenge, so ist die konvexe
    Hülle $\conv(A)\subset\R^d$ eine kompakte Menge.
\end{thLemma}

\begin{proof}
    Seien $n\in\N$ und $v_0,\ldots,v_n$ die verschiedenen Punkte aus $A$
    (wobei o.\,E. $A\neq\emptyset$).
    Nach \cref{gsc:convexhullviaconvexcombinations} gilt dann:
    \[ \conv(\{v_0,\ldots,v_n\}) = 
        \Bigl\{\, \isum^n \lambda_i\,v_i \Mid[\;] \lambda\in(\R[\geq0])^{n+1},
        \; \isum^n \lambda_i = 1    \,\Bigr\}
    \]
    Die Menge 
    $T\defeq\{ \lambda\in [0,1]^{n+1} \Mid \isum^n \lambda_i = 1 \}$
    ist abgeschlossen in $[0,1]^{n+1}$ als Urbild der in $\R$ abgeschlossenen
    Menge $\{1\}$ unter der stetigen Funktion $[0,1]^{n+1}\to\R,\;
    \lambda\mapsto\isum^n \lambda_i$. Weil $[0,1]^{n+1}$ kompakt ist, ist somit
    auch $T$ kompakt. Damit ist die obige konvexe Hülle kompakt als Bild der
    kompakten Menge~$T$ unter der stetigen Abbildung
    \[ T \to \R^d, \quad \lambda\mapsto \isum^n \lambda_i\,v_i . \]
\end{proof}

\begin{thKorollar}[Simplizes sind kompakt]
    \label{gsc:simplicescompact}
    %
    Ist $\sigma$ ein Simplex, so ist $\sigma$ mit der Teilraumtopologie ein
    kompakter topologischer Raum.
\end{thKorollar}

\begin{thSatz}[Homöomorphie zwischen Bällen und konvexen Mengen]
    \label{gsc:convexsethomeodisk}
    %
    Sei $d\in\N$ und $A\subset\R^d$ eine konvexe und kompakte Teilmenge mit
    nicht-leerem Inneren $\setinterior A$. Dann gelten folgende Homöomorphien:
    \[ A\cong D^d \qqtextqq{mit} 
        \setboundary{A} \cong S^{d-1} \subset D^d 
    \]
    (als Einschränkung des linken Homöomorphismus).
\end{thSatz}

Einen Beweis dieses Satzes findet man beispielsweise bei
Bredon\cite[Ch.\,I,\;16.]{bookc:bredon93} oder bei
Munkres\cite[Ch.\,1,\;\S1,\;1.1]{bookc:munkres84}.

\begin{thBeispiel}\label{gsc:bsp:spheretriang}
    Zu $d\in\N[\geq1]$ erhalten wir also eine Triangulierung der Sphäre
    $S^{d-1}$ wie folgt: Sei $\sigma$ ein beliebiges $d$-Simplex. Wir bilden
    dann den zugehörigen simplizialen Komplex und entfernen die einzige
    $d$-dimensionale Seite (also $\sigma$ selbst) und erhalten so einen
    simplizialen Komplex $\Delta(\sigma)\setminus\{\sigma\}$, dessen Polyeder
    nach \cref{gsc:simplicescompact} und \cref{gsc:convexsethomeodisk} zur
    Einheitssphäre~$S^{d-1}$ homöomorph ist. Der Homöomorphismus ist dann
    gerade durch die Zentralprojektion von einem Punkt im Inneren des Simplex
    aus auf die Sphärenoberfläche gegeben (siehe \cref{gsc:fig:spheretriang}).
\end{thBeispiel}

\begin{figure}
    \centering
    \begin{tikzpicture}[scale=3]
        \draw (0,0) -- (1,0) -- ++(120:1) -- cycle;
        \path [name path=m1] (0,0) -- (30:1);
        \path [name path=m2] (1,0) -- ++(150:1);
        \path [name intersections={of=m1 and m2,by=C}];
        \draw let \p1=(C), \n1 = {veclen(\x1,\y1)} in (C) circle [radius=\n1];
        
        \foreach \ang in {0,30,...,330} 
            \draw let \p1=(C), \n1 = {veclen(\x1,\y1)} in
                [->, color=black!40, opacity=0.65, densely dotted]
                (C) -- +(\ang:\n1);
            
    \end{tikzpicture}
    \caption{\cref{gsc:bsp:spheretriang} für $d=2$}
    \label{gsc:fig:spheretriang}
\end{figure}

Weitere Triangulierungen von Sphären erhalten wir (auch mithilfe der obigen
Resultate) aus sogenannten \emph{Kreuzpolytopen}:

\begin{thDef}[Kreuzpolytop]
    Zu $d\in\N$ bezeichnet
    \[ \conv(\{ \pm e_1, \ldots, \pm e_d \}) \;\subset\R^d \]
    das \emph{$d$-dimensionale Kreuzpolytop}.
\end{thDef}

% TODO: Skizzen für d=0..3

Offenbar ist $\conv(\{\alpha_1\,e_1,\ldots,\alpha_d\,e_d\})$ für
$\alpha\in\{-1,1\}^d$ ein Simplex und alle derartigen Simplizes zusammen ergeben
gerade den Rand des entsprechenden Kreuzpolytops. Die Kollektion aller solchen
Simplizes zusammen mit allen Seiten ergibt also einen simplizialen Komplex,
womit wir analog zum letzten Beispiel eine Triangulierung der Sphäre erhalten.
Aufgrund der offensichtlichen Symmetrieeigenschaften der Kreuzpolytope erhalten
wir somit nützliche symmetrische Triangulierungen. 

Wir betrachten noch ein weiteres Beispiel:

\begin{thBeispiel}[Triangulierung des Torus]
    Wir erinnern uns daran, dass wir einen Torus erhalten, wenn wir
    gegenüberliegende Seiten eines Quadrats (in gleicher Richtung) miteinander
    identifizieren.
    \begin{center}
        \newcommand{\myarrow}[1][0]{%
            \tikz\draw[-triangle 90] (0,0) -- (#1:0.01pt);
        }
        \begin{tikzpicture}[%
                scale=2,
                mypoint/.style={shape=circle, inner sep=1.5pt, color=black, fill}
            ]
            \draw [fill=black!20] 
                (0,0) -- (1,0) node [pos=0.55] {\myarrow[0]}
                      -- (1,1) node [pos=0.50] {\myarrow[90]}
                               node [pos=0.55] {\myarrow[90]}
                      -- (0,1) node [pos=0.45] {\myarrow[0]}
                      -- (0,0) node [pos=0.45] {\myarrow[90]}
                               node [pos=0.50] {\myarrow[90]};
        \end{tikzpicture}
    \end{center}
    Diese Beobachtung können wir nun benutzen, um eine Triangulierung des Torus
    zu konstruieren. Dabei müssen wir aber vorsichtig sein, denn zum Beispiel
    stellt 
    \begin{center}
        \begin{tikzpicture}[%
                scale=3,
                mypoint/.style={shape=circle, inner sep=1.5pt, color=black, fill}
            ]
            \draw [fill=black!20] (0,0) rectangle (1,1);
            \draw (0,0) -- (1,1);
            \foreach \mycoord in {(0,0),(0,1),(1,0),(1,1)}
                \node [mypoint] at \mycoord {};
        \end{tikzpicture}
    \end{center}
    eine gültige Triangulierung des Quadrats dar, wenn wir aber die Seiten wie
    zuvor miteinander identifizieren, so bilden die dargestellten Dreiecke
    keinen simplizialen Komplex mehr! (Warum?) Wir benötigen also eine feinere
    Triangulierung des Ausgangsquadrats, zum Beispiel folgende:
    \begin{center}
        \begin{tikzpicture}[%
                mypoint/.style={shape=circle, inner sep=1.5pt, color=black, fill}
        ]
            \foreach \myx in {0,1,2} {
                \foreach \myy in {0,1,2} {
                    \begin{scope}[shift={(\myx,\myy)}]
                        \draw [fill=black!20] (0,0) rectangle (1,1);
                        \draw (0,0) -- (1,1);
                        \foreach \mycoord in {(0,0),(0,1),(1,0),(1,1)}
                        \node [mypoint] at \mycoord {};
                    \end{scope}
                }
            }
        \end{tikzpicture}
    \end{center}
    Identifizieren wir nun die äußeren Seiten, so sehen wir, dass wir eine
    Triangulierung des Torus gefunden haben.\footnote{%
        Es gibt Triangulierungen des Torus mit weniger $2$-Simplizes. Allerdings
        kann man zeigen, dass man mindestens $14$~Dreiecke benötigt, siehe
        z.\,B. Pelletier\cite{www:blog:rip:triangulation}.%
    }
    Im Allgemeinen müssen wir also genau aufpassen, ob eine gegebene
    Triangulierung nach Bildung eines Quotienten immer noch die Bedingungen
    eines simplizialen Komplexes erfüllt. 
\end{thBeispiel}


\chapter{Abstrakte simpliziale Komplexe}
\section{Definition und geometrische Realisierung}
Um nicht an die geometrische Definition von simplizialen Komplexen
im $\R^d$ gebunden zu sein, abstrahieren wir nun von diesem geometrischen Konzept,
um simpliziale Komplexe rein kombinatorisch und unabhängig von euklidischen Räumen
beschreiben zu können.

\begin{thDef}%
    [Abstrakter simplizialer Komplex, abstraktes Simplex, Dimension, Teilkomplex]%\hfill\\
    Ist $V$ eine Menge und $K\subset\pot{V}$, so dass alle Elemente von $K$
    endliche Mächtigkeit haben, so ist $(V,K)$ ein \emph{abstrakter simplizialer
    Komplex}, falls zusätzlich gilt:
    \[ \forall\,F\in K 
        \;\forall\, F'\subset F \colon\; F'\in K
    . \]
    Ein Element $F\in K$ nennen wir \emph{(abstraktes) Simplex} und die
    \emph{Dimension von $F$} ist gegeben durch $\dim(F) \defeq
    \abs{F}-1$.
    Die \emph{Dimension von $K$} ist das Maximum der Dimensionen seiner Simplizes (oder
    $\infty$, falls dieses nicht existiert):
    \[ \dim(K) \defeq \max_{F\in K}\,\dim(F)  \;\in\N\cup\{\infty\}  . \]
    Ist $\abs{V}$ endlich, so sprechen wir von einem \emph{endlichen
    simplizialen Komplex}. Sind $V'\subset V,$ $K'\subset K$ und definiert
    $(V',K')$ selbst einen simplizialen Komplex, so nennen wir $(V',K')$ einen
    \emph{Teilkomplex von $(V,K)$}.
\end{thDef}

In Zukunft notieren wir anstatt des Paars $(V,K)$ oft nur noch $K$ und
nehmen an, dass $V=\bigcup K$ gilt. Außerdem können wir aus jedem geometrischen
simplizialen Komplex~$\Delta$ einen abstrakten gewinnen, indem wir wie folgt
vorgehen: Sei $V \defeq V(\Delta)$ die Eckenmenge von $\Delta$ und $K$ wie folgt
gegeben:
\[ K \defeq \bigl\{ V_\sigma 
    \Mid V_\sigma \text{ ist Eckenmenge eines Simplex $\sigma\in\Delta$} \bigr\}
. \]
Dann ist $(V,K)$ klarerweise ein abstrakter simplizialer Komplex, welchen wir
eine \emph{geometrische Realisierung von $\Delta$} nennen. (Siehe
\cref{asc:fig:geomrealization}.) Das Polyeder $\polyeder\Delta$ bezeichnen wir
dann auch als ein \emph{Polyeder von $K$}.

\begin{figure}
    \centering
    \begin{tikzpicture}[%
        mypoint/.style={shape=circle, inner sep=1.5pt, color=black, fill}
    ]
        \begin{scope}[scale=0.90]
            \coordinate (1) at (0,0);
            \coordinate (2) at (1,-1);
            \coordinate (3) at (2,0);
            \coordinate (4) at (3,1);
            \coordinate (5) at (2,2);
            \coordinate (6) at (0.5,1.5);
            
            \draw (1) -- (2) -- (3) -- cycle;
            \draw [fill=black!20] (3) -- (4) -- (5) -- cycle;
            
            \foreach \n/\ang in 
                {1/left,2/above,3/below right,4/right,5/left,6/above}
                \node [mypoint,label=\ang:$\n$] at (\n) {};
        \end{scope}

        \node [align=left] at (8,1) {%
            $\displaystyle
            \bigl\{\; \emptyset, \{1\}, \{2\}, \{3\}, \{4\}, \{5\}, \{6\},$    \\
            $\displaystyle\hphantom{\bigl\{\;}
                    \{1,2\}, \{2,3\}, \{1,3\},$                     \\
            $\displaystyle\hphantom{\bigl\{\;}
                    \{3,4\}, \{4,5\}, \{3,5\}, \{3,4,5\}
            \;\bigr\}$%
        };
    \end{tikzpicture}
    \caption{Geometrische Realisierung (links) und zugehöriger abstrakter
    simplizialer Komplex (rechts)}
    \label{asc:fig:geomrealization}
\end{figure}

\begin{thBeispiel}[Graphen als abstrakte simpliziale Komplexe]
    Jeder (einfache, ungerichtete) Graph $(\tilde V,E)$ mit
    Eckenmenge~$\tilde V$ und Kantenmenge~$E$ definiert einen
    eindimensionalen simplizialen Komplex~$(V,K)$: Setze $V\defeq\tilde V$ und
    $K\defeq \{ A \Mid A\subset \{u,v\}\in E \}$. 
    (Siehe \cref{asc:fig:geomrealization}.)
\end{thBeispiel}

\begin{figure}
    \centering
    \begin{tikzpicture}[%
            mypoint/.style={shape=circle, inner sep=1.5pt, color=black, fill}
    ]
    \coordinate (1) at (0,0);
    \coordinate (2) at (2,0);
    \coordinate (3) at (2,1);
    \coordinate (4) at (2,2);
    \coordinate (5) at (0,2);
    \coordinate (6) at (1,1);

    \draw (1) \foreach \n in {2,...,6,1,5} { -- (\n) } (6) -- (3);
    \foreach \n/\ang in 
        {1/left,2/right,3/right,4/right,5/left,6/above right}
        \node [mypoint,label=\ang:$\n$] at (\n) {};
        
    \node [align=left] at (5.5,1) {%
        $\displaystyle
            \tilde V = \{ 1,2,3,4,5,6 \}$       \\
        $\displaystyle
            E = \bigl\{\; 
                \{1,2\}, \{2,3\},$              \\
        $\displaystyle\quad
                \{3,4\}, \{4,5\}, \{5,1\},$     \\
        $\displaystyle\quad
                \{1,6\}, \{3,6\}, \{5,6\}
        \;\bigr\}$
    };
    
    \node [align=left] at (10.5,1) {%
        $\displaystyle
        \bigl\{\; \emptyset, \{1\}, \{2\}, \{3\},$      \\
        $\displaystyle\hphantom{\bigl\{\;}
                \{4\}, \{5\}, \{6\},$                   \\
        $\displaystyle\hphantom{\bigl\{\;}
                \{1,2\}, \{2,3\},$                      \\
        $\displaystyle\hphantom{\bigl\{\;}
                \{3,4\}, \{4,5\}, \{5,1\},$             \\
        $\displaystyle\hphantom{\bigl\{\;}
                \{1,6\}, \{3,6\}, \{5,6\}
        \;\bigr\}$
    };
    \end{tikzpicture}
    \caption{Graph (links), Graph $(\tilde V,E)$ abstrakt (mittig) 
        und zugehöriger abstrakter simplizialer Komplex (rechts)}
    \label{asc:fig:graphtocomplex}
\end{figure}

Man sieht relativ einfach ein, dass jeder endliche simpliziale Komplex $(V,K)$ 
eine geometrische Realisierung im $\R^{\abs{V}-1}$ hat, siehe beispielsweise
Matou\v sek\cite[Ch.\,1,\;1.5]{bookc:matousek03}, oder in einer allgemeineren
Variante Munkres\cite[Ch.\,1,\;\S3,\;3.1]{bookc:munkres84}. Stattdessen
verweisen wir hier auf \cref{asc:geomrealization}, welcher eine schärfere Aussage
liefert.


\section{Dimension geometrischer Realisierungen}
Wir wollen nun folgenden Satz beweisen, welcher uns eine obere Schranke für die
nötige Dimension $d\in\N$ gibt, für welche es zu einem abstrakten simplizalen
Komplex eine geometrische Realisierung im $\R^d$ gibt.

\begin{thSatz}%
    [Satz über die maximal nötige Dimension einer geometrischer Realisierung]
    \label{asc:geomrealization}
    \hfill\\
    %
    Sei $K$ ein endlicher abstrakter simplizaler Komplex der Dimension 
    $\dim(K)=d\in\N$.
    Dann besitzt $K$ eine geometrische Realisierung im $\R^{2d+1}$.
\end{thSatz}

Zum Beweis dieses Satzes benötigen wir noch einige Hilfsmittel. 

\begin{thLemma}[Hinreichende Bedingung für geometrische Realisierung]
    \label{asc:suffprelimgeomrealization}
    %
    Seien $d\in\N,\; (V,K)$ ein abstrakter simplizialer Komplex und
    $f\colon V\to\R^d$ eine injektive Abbildung mit folgender Eigenschaft:
    \[ \forall\,F,G\in K\colon \; f(F\cup G) \text{ ist eine Menge affin
        unabhängiger Vektoren}
    . \]
    Dann ist 
    \[ \bigl\{ \conv\bigl(f(F)\bigr) \cMid[\;]\big F\in K \bigr\} \]
    eine geometrische Realisierung von $K$ im $\R^d$.
\end{thLemma}

\begin{proof}
    Es ist nur zu zeigen, dass die zweite Bedingung aus \cref{gsc:def:gsc}
    erfüllt ist (-- der Rest ist klar). Seien dazu $F,G\in K$. 
    Da $f(F\cup G)$ nach Voraussetzung affin unabhängig ist, definiert 
    $\sigma \defeq \conv\bigl(f(F\cup G)\bigr)$ ein Simplex im $\R^d$. 
    Außerdem gilt offenbar $f(F), f(G)\subset f(F\cup G)$, weshalb
    $\conv\bigl(f(F)\bigr)$ und 
    $\conv\bigl(f(G)\bigr)$ Seiten von $\sigma$
    sind. Da die Seiten eines Simplex nach \cref{gsc:complexofsimplex} einen
    simplizialen Komplex bilden, ist auch der Schnitt dieser beiden Seiten 
    eine Seite von $\sigma$, etwa $\conv\bigl(f(T)\bigr)$ für 
    $T\subset F\cup G$. Dann gilt also:
    \[ \conv\bigl(f(F)\bigr) \cap \conv\bigl(f(G)\bigr) 
        = \conv\bigl(f(T)\bigr)
    . \]
    Da $f(F\cup G)$ affin unabhängig ist, sieht man schnell ein, dass das Bilden
    der konvexen Hülle über Teilmengen davon eine injektive Operation ist.
    Zusammen mit der gegebenen Injektivität von~$f$ ergibt sich
    $T = F \cap G$, und da letztere Menge sicherlich auch in~$K$ enthalten ist, 
    sind wir fertig.
    \\
\end{proof}

Um nun Abbildungen zu finden, auf die wir \cref{asc:suffprelimgeomrealization}
anwenden können, bedienen wir uns der sogenannten \emph{Momentenkurve} und ihrer
nützlichen Eigenschaften:

\begin{thDef}[Momentenkurve]
    Zu $d\in\N$ ist die \emph{Momentenkurve im $\R^d$} gegeben durch:
    \[  \R \to \R^d, \quad x\mapsto (x,x^2,\ldots,x^d)  . \]
\end{thDef}

\begin{thLemma}[Eigenschaften der Momentenkurve]
    \label{asc:momentumcurveprop}
    %
    Sei $d\in\N$ und $\gamma$ die Momentenkurve in $\R^d$. Dann hat $\gamma$
    folgende Eigenschaften:
    \begin{enumerate}[a)]
        \item
            Jede (affine) Hyperebene im $\R^d$ schneidet $\gamma$ 
            in maximal~$d$ Punkten.
        \item\label{asc:momentumcurveprop:pointsaffinlyindependent}
            Je $d+1$ Punkte aus $\Image(\gamma)$ sind affin unabhängig.
        \item
            Schneidet eine (affine) Hyperebene $\gamma$ in $d$ verschiedenen 
            Punkten, so wechselt $\gamma$ an jeder dieser Stellen von einem
            durch die Hyperebene gegebenen Halbraum zum anderen.
    \end{enumerate}
\end{thLemma}

\begin{proofsketch}
    Indem man eine die Hyperebene definierende Gleichung und die besondere Form
    der Momentenkurve betrachtet, sieht man, dass sich die Aussagen über
    Schnittpunkte auf algebraische Argumente über Nullstellen von Polynomen
    zurückführen lassen. (Genaueres findet man bei 
    Matou\v sek\cite[Ch.\,1,\;1.6.4]{bookc:matousek03}.)
    \\
\end{proofsketch}

Wir haben nun alles beisammen, um die Hauptaussage dieses Abschnitts beweisen zu
können:
\begin{proof}[Beweis von \cref{asc:geomrealization}]
    Sei also $K$ ein endlicher simplizialer Komplex mit $\dim(K)=d\in\N$.
    Seien außerdem $m\in\N$ und $x_0,\ldots,x_m$ die (verschiedenen) Punkte
    aus $\bigcup K$. Bezeichne weiter $\gamma$ die Momentenkurve im $\R^{2d+1}$.
    Definiere dann eine Abbildung
    $f\colon \bigcup K\to\R^{2d+1}$ für $i\in\setZeroto m$ durch 
    \[ f(x_i) \defeq \gamma(i) 
    . \]
    Es ist $f$ offenbar injektiv und außerdem gilt für alle $F,G\in K$:
    \[ \abs{F\cup G} \leq \abs{F}+\abs{G} \leq (d+1)+(d+1) = 2d+2  . \]
    Also sind nach \mycref{asc:momentumcurveprop:pointsaffinlyindependent}
    die Vektoren in $f(F\cup G)$ affin unabhängig und wir können
    \cref{asc:suffprelimgeomrealization} anwenden.
    \\
\end{proof}


\section{Zusammenhang zu partiell geordneten Mengen}
Zur Erinnerung geben wir kurz die folgende Definition:

\begin{thDef}[Partiell geordnete Menge]
    Eine \emph{partiell geordnete Menge $(P,\leq)$} besteht aus einer Menge~$P$
    und einer Relation~$\leq$ auf $P$ mit den folgenden Eigenschaften:
    \begin{enumerate}[a)]
        \item
            Reflexivität:\quad $\forall\, x\in P\colon\; x \leq x$
        \item
            Transitivität:\quad $\forall\, x,y,z\in P\colon 
            (x\leq y \,\wedge\, y\leq z) \implies x \leq z$
        \item
            Anti-Symmetrie:\quad $\forall\, x,y\in P\colon
            (x\leq y \,\wedge\, y\leq x) \implies x = y$
    \end{enumerate}
\end{thDef}

Wir können aus jeder endlichen partiell geordneten Menge einen endlichen 
simplizialen Komplex bilden und umgekehrt.
Folgende Definitionen legen fest, wie wir dies tun wollen:

\begin{thDef}[Ordnungskomplex]
    Ist $(P,\leq)$ eine endliche partiell geordnete Menge, so definieren wir den
    \emph{Ordnungskomplex $\Delta(P)$} wie folgt:
    \[ \Delta(P) \defeq \bigl\{ \{x_1,\ldots,x_k\} \subset P \cMid[\;]\big 
        k\in\N,\; x_1\leq \dots \leq x_k \bigr\}
    . \]
    Die Simplizes von $\Delta(P)$ sind also alle (endlichen) Ketten aus $P$.
\end{thDef}

\begin{thDef}[Partielle Ordnung auf simplizialem Komplex]
    Ist $K$ ein endlicher simplizialer Komplex, so wird
    $K\setminus\{\emptyset\}$ durch die Inklusionsrelation~\enquote{$\subset$} 
    partiell geordnet. Wir bezeichnen diese zu $K$ assoziierte (endliche) 
    partiell geordnete Menge mit $P(K)$.
\end{thDef}


\begin{figure}[b]
    \centering
    \begin{tikzpicture}[%
        mypoint/.style={shape=circle, inner sep=1.5pt, color=black, fill}
    ]
        \node [align=left] at (0,1) {%
            $\displaystyle
            \bigl\{\; \emptyset, \{1\}, \{3\}, \{5\}, \{15\}$       \\
            $\displaystyle\hphantom{\bigl\{\;}
                    \{1,3\}, \{1,5\}, \{1,15\}$                     \\
            $\displaystyle\hphantom{\bigl\{\;}
                    \{3,15\}, \{5,15\},$                            \\
            $\displaystyle\hphantom{\bigl\{\;}
                    \{1,3,15\}, \{1,5,15\}
            \;\bigr\}$
        };
        
        \begin{scope}[shift={(7,0)}]
            \coordinate (1) at (0,0);
            \coordinate (3) at (1,1);
            \coordinate (15) at (0,2);
            \coordinate (5) at (-1,1);
            
            \draw [fill=black!20] (1) -- (3) -- (15) -- (5) -- cycle;
            \draw (1) -- (15);
            
            \foreach \n/\ang in 
                {1/below,3/right,15/above,5/left}
                \node [mypoint,label=\ang:$\n$] at (\n) {};
        \end{scope}
    \end{tikzpicture}
    \caption{Ordnungskomplex $K\defeq \Delta(T)$ (links) zur partiell geordneten 
        Menge $(T,\mkern2mu|\mkern1mu)$ mit $T = \{1,3,5,15\}$
             (wobei $|$ die Teilerrelation bezeichnet) und eine geometrische
             Realisierung (rechts)}
    \label{asc:fig:ordercomplex}
\end{figure}

\begin{figure}[p]
    \newcommand\X[1]{$\{#1\}$}
    \centering
    \begin{tikzpicture}[%
        node distance=2.2cm
    ]
        \node (1) at (0,0)      {\X 1};
        \node (3)  [right of=1] {\X 3};
        \node (5)  [right of=3] {\X 5};
        \node (15) [right of=5] {\X{15}};
        
        \node (1-5)  [above of=1]  {\X{1,5}};
        \node (1-3)  [left of=1-5] {\X{1,3}};
        \node (1-15) [above of=3]  {\X{1,15}};
        \node (3-15) [above of=5]  {\X{3,15}};
        \node (5-15) [above of=15] {\X{5,15}};
        
        \node (1-3-15) [above of=1-5] {\X{1,3,15}};
        \node (1-5-15) [above of=3-15] {\X{1,5,15}};
        
        \draw (1) -- (1-3)     (3) -- (1-3)     (5) -- (1-5)    (15) -- (1-15)
              (1) -- (1-5)     (3) -- (3-15)    (5) -- (5-15)   (15) -- (3-15)
              (1) -- (1-15)                                     (15) -- (5-15)
              
              (1-3) -- (1-3-15)
              (1-5) -- (1-5-15)
              (1-15) -- (1-3-15)
              (1-15) -- (1-5-15)
              (3-15) -- (1-3-15)
              (5-15) -- (1-5-15)
        ;
        
        \draw [->, very thick] (9,-0.3) -- node [right=7pt,scale=1.5] {$\subset$} (9,5);
    \end{tikzpicture}
    \caption{Hasse-Diagramm zur partiell geordneten Menge $P(K)$, d.\,h.
        derjenigen partiell geordneten Menge, die wir dem Komplex $K=\Delta(T)$
        zuordnen (s.\,a. \cref{asc:fig:ordercomplex})}
    \label{asc:fig:hasse}
\end{figure}

\begin{figure}[p]
    \hspace*{-1.6cm}
    \begin{tikzpicture}[%
        mypoint/.style={shape=circle, inner sep=1.5pt, color=black, fill}
    ]
        \node [align=left] at (0,1) {%
            $\displaystyle
            \smash{\Bigl\{}\; \emptyset, 
                \bigl\{  \{1\}                        \bigr\}, 
                \bigl\{  \{3\}                        \bigr\}, 
                \bigl\{  \{5\}                        \bigr\}, 
                \bigl\{  \{15\}                       \bigr\},
                \bigl\{  \{1,3\}                      \bigr\},
                \bigl\{  \{1,5\}                      \bigr\},
                \bigl\{  \{1,15\}                     \bigr\},
                \bigl\{  \{3,15\}                     \bigr\},
                \bigl\{  \{5,15\}                     \bigr\},        
                \bigl\{  \{1,3,15\}                   \bigr\},
                \bigl\{  \{1,5,15\}                   \bigr\},  $\\[2pt]$\displaystyle \;\;
                \bigl\{  \{1\},\{1,3\}                \bigr\},
                \bigl\{  \{1\},\{1,5\}                \bigr\},
                \bigl\{  \{1\},\{1,15\}               \bigr\},
                \bigl\{  \{3\},\{1,3\}                \bigr\},  
                \bigl\{  \{3\},\{3,15\}               \bigr\},
                \bigl\{  \{5\},\{1,5\}                \bigr\},
                \bigl\{  \{5\},\{5,15\}               \bigr\},  $\\[2pt]$\displaystyle \;\;
                \bigl\{  \{15\},\{1,15\}              \bigr\},
                \bigl\{  \{15\},\{3,15\}              \bigr\},
                \bigl\{  \{15\},\{5,15\}              \bigr\},  
                \bigl\{  \{1,3\},\{1,3,15\}           \bigr\},
                \bigl\{  \{1,15\},\{1,3,15\}          \bigr\},
                \bigl\{  \{3,15\},\{1,3,15\}          \bigr\},  $\\[2pt]$\displaystyle \;\;
                \bigl\{  \{1,5\},\{1,5,15\}           \bigr\},
                \bigl\{  \{1,15\},\{1,5,15\}          \bigr\},  
                \bigl\{  \{5,15\},\{1,5,15\}          \bigr\},  \quad
                \bigl\{  \{1\},\{1,3\},\{1,3,15\}     \bigr\},
                \bigl\{  \{1\},\{1,5\},\{1,3,15\}     \bigr\},  $\\[2pt]$\displaystyle \;\;
                \bigl\{  \{1\},\{1,15\},\{1,3,15\}    \bigr\},
                \bigl\{  \{3\},\{1,3\},\{1,3,15\}     \bigr\},
                \bigl\{  \{3\},\{3,15\},\{1,3,15\}    \bigr\},
                \bigl\{  \{5\},\{1,5\},\{1,5,15\}     \bigr\},  $\\[2pt]$\displaystyle \;\;
                \bigl\{  \{5\},\{5,15\},\{1,5,15\}    \bigr\},
                \bigl\{  \{15\},\{1,15\},\{1,5,15\}   \bigr\},
                \bigl\{  \{15\},\{3,15\},\{1,5,15\}   \bigr\},
                \bigl\{  \{15\},\{5,15\},\{1,5,15\}   \bigr\}
            \;\smash{\Bigr\}}$
        };
        
        \begin{scope}[shift={(0,-6.5)},scale=2]
            \coordinate (1)  at (0,0);
            \coordinate (3)  at (30:2);
            \coordinate (15) at (0,2);
            \coordinate (5)  at (150:2);
            
            \coordinate (1-3)  at ($ (1)!0.5!(3) $);
            \coordinate (1-5)  at ($ (1)!0.5!(5) $);
            \coordinate (3-15) at ($ (3)!0.5!(15) $);
            \coordinate (5-15) at ($ (5)!0.5!(15) $);
            \coordinate (1-15) at ($ (1)!0.5!(15) $);
            
            \coordinate (1-3-15) at (0.5,1);
            \coordinate (1-5-15) at (-0.5,1);
            
            \draw [fill=black!20] (1) -- (3) -- (15) -- (5) -- cycle;
            \draw (1) -- (15);
            
            \foreach \n/\ang in 
                {1/below,3/right,15/above,5/left}
                \node [mypoint,label=\ang:$\{\n\}$] at (\n) {};
            \foreach \n/\m/\ang in 
                {1/3/-45,1/5/-135,3/15/45,5/15/135}
                \node [mypoint,label=\ang:{$\{\n,\m\}$}] at (\n-\m) {};
            
            \foreach \mycoord in 
                {1-15,1-3-15,1-5-15}
                \node [mypoint] at (\mycoord) {};
                
            \foreach \mycoord in
                {1,3,15,1-3,1-15,3-15}
                \draw (1-3-15) -- (\mycoord);
            \foreach \mycoord in
                {1,5,15,1-5,1-15,5-15}
                \draw (1-5-15) -- (\mycoord);
                
            \node (L1-15)   at (-1,-0.1) {$\{1,15\}$};
            \node (L1-3-15) at (2.5,1.6) {$\{1,3,15\}$};
            \node (L1-5-15) at (-2.5,1.6) {$\{1,5,15\}$};
            
            \begin{scope}[color=black!40, opacity=0.65, very thin]
                \draw (L1-15) -- ($ (1-15)+(-0.05,-0.05) $);
                \draw (L1-3-15) -- ($ (1-3-15)+(0.05,0.03) $);
                \draw (L1-5-15) -- ($ (1-5-15)+(-0.05,0.03) $);
            \end{scope}
        \end{scope}
    \end{tikzpicture}
    \caption{Ordnungskomplex (oben, noch einmal ganz ausführlich) 
             zur partiell geordneten Menge $P(K)$ aus
             \cref{asc:fig:hasse} und geometrische Realisierung (unten)}
    \label{asc:fig:barycentricsubdivision}
\end{figure}





\chapter{Quotienten von simplizialen Komplexen}
Wir wollen nun zeigen, dass sich Quotienten von Polyedern nach kontraktiblen
Teilpolyedern (induziert von einem Teilkomplex) in ihrem Homotopietyp nicht vom
Ausgangsraum unterscheiden. Da wir aus aus dem letzten Semester\footnote{%
    Siehe Löh\cite[Kap.\,3,\;3.2,\;3.24\;u.\,3.28]{lecnotes:loeh:at1}, bzw. die
    entsprechende Übungsaufgabe. Alternativ findet man einen Beweis auch bei
    Hatcher\cite[Ch.\,0,\;0.17]{bookc:hatcher02}.%
}
schon allgemeiner wissen, dass dies gilt, falls die Inklusion des Teilraums eine
Kofaserung ist, zeigen wir also folgende Aussage, woraus wir direkt das
anschließende gewünschte Korollar erhalten:

\begin{thLemma}[Inklusion eines Teilkomplexes ist Kofaserung]
    \label{quot:inclusioncofib}
    %
    Sei $K$ ein endlicher simplizialer Komplex und $L\subset K$ ein
    Teilkomplex.  Dann ist die Inklusion zugehöriger Polyeder (einer
    geometrischen Realisierung) $\polyeder L \hookrightarrow \polyeder K$ eine
    Kofaserung.
\end{thLemma}

\begin{thKorollar}%
    [Quotienten von simplizialen Komplexen nach kontraktiblen Teilkomplexen]
    %
    Sei $K$ ein endlicher simplizialer Komplex und $L\subset K$ ein
    Teilkomplex, für den $\polyeder L$ kontraktibel ist. Dann sind $\polyeder K$
    und $\polyeder K\mkern2mu/\mkern2mu\polyeder L$ homotopieäquivalente Räume.
\end{thKorollar}

Bevor wir \cref{quot:inclusioncofib} zeigen können, brauchen wir zuerst noch
ein Hilfsresultat, das aber auch für sich allein genommen interessant ist:

\begin{thLemma}[Abgeschlossenheit von Teilkomplexen]
    \label{quot:subcomplexclosed}
    %
    Sei $\Delta$ ein (geometrischer) simplizialer Komplex und sei
    $\Delta'\subset\Delta$ ein Teilkomplex. Dann ist $\polyeder{\Delta'}$
    abgeschlossen in $\polyeder\Delta$.
\end{thLemma}

\begin{proof}
    Sei $\sigma\in\Delta$. Dann gilt:
    \[ \polyeder{\Delta'}\cap\sigma 
        = \bigcup\, \bigl\{ \sigma'\in\Delta(\sigma) \cMid\big \sigma'\in\Delta'
        \bigr\}
    , \]
    d.\,h. $\polyeder{\Delta'}\cap\sigma$ ist die Vereinigung aller Seiten von
    $\sigma$, welche in $\Delta'$ enthalten sind. Jedes der obigen $\sigma'$ ist
    aber abgeschlossen in $\sigma$ (als kompakte Teilmenge eines Hausdorffraums)
    und damit ist die endliche Vereinigung abgeschlossener Mengen auf der rechten
    Seite ebenfalls abgeschlossen in $\sigma$. Da $\sigma$ beliebig war, ist
    $\polyeder{\Delta'}$ also abgeschlossen in $\Delta$ nach Definition der
    Topologie auf $\polyeder\Delta$.
    \\
\end{proof}

%\begin{thLemma}[Stetigkeit einer Abbildung aus einem Polyeder heraus]
%    Ist $\Delta$ ein (geometrischer) simplizialer Komplex, $X$ ein topologischer
%    Raum und $f\colon\polyeder\Delta\to X$ eine Abbildung, so gilt:
%    \[ f\text{ stetig} 
%        \qiffq \forall\,\sigma\in\Delta\colon\; f\vert_\sigma \text{ stetig}
%    \]
%\end{thLemma}
%
%Der Beweis ist eine einfache Übung und ergibt sich direkt aus der Topologie auf
%$\polyeder\Delta$. Wir widmen uns nun dem Beweis von \cref{quot:inclusioncofib}.

\medskip
\begin{proof}[Beweis von \cref{quot:inclusioncofib}]
    \belowpdfbookmark{Beweis von Lemma \ref{quot:inclusioncofib}}{proofofinclusioncofib}
    %
    Sei $K$ ein endlicher simplizialer Komplex und $L\subset K$ ein
    Teilkomplex. Um die Notation übersichtlicher zu halten, bezeichnen wir die
    Polyeder von $K$ bzw. $L$ mit $X$ bzw. $A$. Weiter bezeichne im Folgenden
    $I\defeq[0,1]$ das Einheitsintervall.

    Da $A$ nach \cref{quot:subcomplexclosed} abgeschlossen in $X$ ist, genügt es
    zu zeigen, dass $(A\times I)\cup(X\times\{0\})$ ein Retrakt von $X\times I$
    ist, d.\,h. dass es eine stetige Abbildung
    \begin{gather*}
        r\colon X\times I \to (A\times I)\cup(X\times\{0\}) 
        \qquad\text{mit} 
        \\ 
        r \circ \bigl( (A\times I)\cup(X\times\{0\}) \hookrightarrow
        X\times I \bigr) = \id_{(A\times I)\cup(X\times\{0\})}
    \end{gather*}
    gibt.\footnote{%
        Man kann zeigen, dass die Bedingung, dass $A$ abgeschlossen ist, nicht
        nötig ist, dass wir also auf \cref{quot:subcomplexclosed} hätten
        verzichten können. Der Beweis dafür ist aber recht kompliziert, wie man
        beispielsweise bei Hatcher\cite[Appendix,\,A.18]{bookc:hatcher02}
        nachlesen kann.%
    }
    Dazu gehen wir wie folgt vor:

    \pagebreak[2]
    Zuerst betrachten wir den Fall, dass $X$ ein Simplex und $A$ der Rand
    des Simplex ist. Sei $n\defeq\dim(X)$. Für $n=-1$ gibt es nichts zu tun und
    für $n=0$ ist $A=\emptyset$, also können wir die Projektion (auf
    $X\times\{0\}$) als Retraktion nehmen. Sei also nun $n\geq1$.
    Dann wissen wir nach \cref{gsc:convexsethomeodisk}, dass es
    einen Homöomorphismus 
    $f\colon X\xrightarrow{\smash{\raisebox{-2pt}{$\sim$}}} D^n$ 
    gibt, welcher $A$ homöomorph auf $S^{n-1}\subset D^n$ abbildet.
    Wir betrachten dann folgendes Diagramm:
    \begin{equation*}
        \begin{xy}
            \xymatrix@C=1.5cm{
                X\times I \ar@{-->}[r] \ar[d]^{f\times\id_I}
                & (A\times I)\cup(X\times\{0\})    
                \\
                D^n\times I \ar[r] 
                & (S^{n-1}\times I)\cup(D^n\times\{0\}) \ar[u]^g
            }
        \end{xy}
    \end{equation*}
    Die untere horizontale Abbildung sei eine Retraktion von $D^n\times I$ auf
    $(S^{n-1}\times I)\cup(D^n\times\{0\})$, welche existiert, da
    $S^{n-1}\hookrightarrow D^n$ eine Kofaserung ist.\footnote{%
        Oder man nehme eine explizite Retraktion zum Beispiel von
        tom Dieck\cite[Ch.\,2,\;2.3.5]{bookc:tomdieck08}, welche bei diesem
        gerade zum Beweis dieser letzten Aussage benutzt wird.%
    }
    Für $g$ nehmen wir die
    verklebte Abbildung 
    \[ g \defeq \bigl(f\vert_A^{-1} \times\id_I\bigr) \cup_{S^{n-1}\times\{0\}}
        \bigl( f^{-1}\times\const_0 \bigr)
    , \]
    wobei $S^{n-1}\times\{0\}$ abgeschlossen in $D^n\times\{0\}$ und somit $g$
    stetig ist. Wir erhalten also die gestrichelte Abbildung, die gesuchte
    Retraktion, indem wir das Diagramm unten herum durchlaufen.

    Nun zum allgemeinen Fall. Ohne Einschränkung gelte 
    $\emptyset\neq L \subsetneq K$. Ist $\Delta$ die geometrische Realisierung
    von $K$ zu welchem $X$ das Polyeder ist, dann gilt:
    \[ X \times I 
        = \Bigl(\, \bigcup_{\sigma\in\Delta} \sigma \mkern2mu\Bigr) \times I
        = \bigcup_{\sigma\in\Delta} (\sigma\times I)
    . \]
    Betrachte dann ein Simplex maximaler Dimension aus $K\setminus L$, bzw. das
    zugehörige geometrische Simplex~$\sigma$. Sei $r'$ dann eine Retraktion wie
    aus dem vorherigen Abschnitt, welche wir mit der Identität auf
    $\bigcup_{\sigma'\in\Delta\setminus\{\sigma\}} (\sigma'\times I)$ verkleben
    können. Da es nur endlich viele solcher Simplizes~$\sigma$ gibt, können wir
    dies der Reihe nach mit allen derartigen Simplizes durchführen bis es keine
    Simplizes aus $K\setminus L$ mehr gibt, welche noch \enquote{aufgedickt}
    sind. Dieser Prozess liefert letztlich also eine Retraktion~$r$ wie
    gewünscht.
    \\
\end{proof}


\nocite{bookc:matousek03}

\appendix
\bibliographystyle{plaindin}
\bibliography{bibsources}

\end{document}





