
%\chapter{Vorwort, Notation und Konventionen}
\chapter{Vorwort und Notation}
Um topologische und kombinatorische Aspekte der Mathematik zusammenzuführen,
benötigen wir eine passende Schnittstelle, die es uns erlaubt, topologische
Räume mit kombinatorischen Mitteln zu untersuchen und umgekehrt aus
kombinatorischen Strukturen topologische Räume zu bauen. Diese Schnittstelle
stellen die sogenannten \emph{simplizialen Komplexe} dar, welche wir im
Folgenden einführen werden. Zunächst in Form geometrischer simplizialer
Komplexe, von welchen wir geeignet abstrahieren werden, um simpliziale Komplexe
in purer Mengentheorie formulieren zu können. Wir werden außerdem betrachten,
wie beide Konzepte zusammenhängen und was simpliziale Komplexe mit partiell
geordneten Mengen zu tun haben. Zuletzt betrachten wir Quotienten simplizialer
Komplexe nach kontraktiblen Teilräumen.


\bigskip
In diesem Skript wird folgende Notation verwendet:
\begin{itemize}
    \item
        Sowohl $\subset$ als auch $\subseteq$ stehen für: enthalten oder gleich.
        Echt enthalten wird durch $\subsetneq$ gekennzeichnet.
    
    \item
        Die \emph{Natürlichen Zahlen $\N$} beginnen mit $0$.
        
    \item
        Zu einer Menge $X$ bezeichnet $\pot X$ die Potenzmenge von $X$.

    \item % TODO: v  bessere Formulierung
        Der $n$-dimensionale Einheitsball wird mit $D^n$ bezeichnet und
        sein Rand $\setboundary{D^n}$, 
        die $(n{-}1)$-dimensionale Einheitssphäre, mit $S^{n-1}$.
\end{itemize}










