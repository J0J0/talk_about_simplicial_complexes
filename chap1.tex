\chapter{Geometrische simpliziale Komplexe}
\section{Definition und erste Eigenschaften} % TODO: besserer Name?
Oftmals ist es hilfreich, einen topologischen Raum in kleinere Bausteine zu
unterteilen. Eine typische solche Unterteilungsmöglichkeit sind \emph{CW-Komplexe}, welche
aber für kombinatorische Anwendungen noch zu allgemein sind.\footnote{%
    In der Tat kann man zeigen, dass es CW-Komplexe gibt, welche sich nicht
    durch die nachfolgenden Konstruktionen beschreiben lassen. Dass dies sogar
    schon mit einem zweidimensionalen CW-Komplex möglich ist, zeigt
    beispielsweise Bognár.\cite{artcle:bognar77}%
}
Wir interessieren uns daher dafür, wie wir topologische Räume beschreiben
können, die aus kombinatorisch einfacher zu beschreibenden Bausteinen aufgebaut
sind. Namentlich werden dies \emph{Strecken, Dreiecke, Tetraeder} und
höherdimensionale Objekte desselben Typus sein, welche mathematisch durch
sogenannte \emph{Simplizes} beschrieben werden können. Dazu benötigen wir
zunächst die folgenden Begriffe:

\begin{thDef}[Affin (un-)abhängig]
    Sind $n,d\in\N$ und $v_0,\ldots,v_n\in\R^d$, so bezeichnen wir diese
    Vektoren als \emph{affin abhängig}, falls es Elemente
    $\lambda_0,\ldots,\lambda_n \in\R$ mit folgenden drei Eigenschaften gibt:
    \[ \{\lambda_0,\ldots,\lambda_n\} \neq \{0\}, \quad \isum^n \lambda_i = 0
    \qundq \isum^n \lambda_i\,v_i = 0 \in\R^d.\] 
    Falls es solche Zahlen nicht gibt, so bezeichnen wir die Vektoren als
    \emph{affin unabhängig}.
\end{thDef}

\begin{thLemma}[Äquivalente Bedingung für affine Unabhängigkeit]
    \label{gsc:iffaffinlyindependet}
    %
    Seien $n,d\in\N$. Dann sind $v_0,\ldots,v_n\in\R^d$ genau dann affin
    unabhängig, wenn für ein beliebiges $k\in\setZeroto{n}$ die Familie
    %\[ \bigl\{ v_i - v_k \Mid i \in\setZeroto{n}\setminus\{k\} \bigr\} \] 
    \[  (v_i - v_k)_{i \in\setZeroto{n}\setminus\{k\}} \] 
    % TODO: ^  vielleicht doch besser auf v_0 fixieren!?
    von Vektoren im $\R^d$ linear unabhängig ist.
\end{thLemma}

% TODO: Beweis

Aus \cref{gsc:iffaffinlyindependet} folgt insbesondere, dass für festes $d\in\N$
maximal $d+1$ Vektoren im $\R^d$ affin unabhängig sein können.

\begin{thDef}[Konvexe Menge, konvexe Hülle, Konvexkombination]
    Sei $d\in\N$. Eine Teilmenge $A\subset\R^d$ heißt \emph{konvex}, wenn
    für alle $x,y\in A$ auch die (lineare) Verbindungsstrecke
    \[ \{ tx + (1-t)\,y \Mid t\in[0,1] \} \]
    in~$A$ enthalten ist.

    \newpage
    \noindent
    Für eine beliebige Teilmenge $A'\subset\R^d$ definiert
    \[ \conv(A') \defeq 
        \bigcap \,\bigl\{ A\subset\R^d \Mid A'\subset A\text{ konvex} \bigr\}
    \]
    die \emph{konvexe Hülle von $A'$}.
    
    \noindent
    Ist außerdem $n\in\N$ und sind $v_0,\ldots,v_n\in\R^d$ beliebig sowie
    $\lambda_0,\ldots,\lambda_n\in\R[\geq0]$ mit $\isum^n\lambda_i = 1$, so
    bezeichnet man $\isum^n \lambda_i\,v_i$ als \emph{Konvexkombination der
    $v_i$}.
\end{thDef}

\begin{thBemerkung} \label{gsc:convexhullviaconvexcombinations}
    Mithilfe einfacher Argumente zeigt man außerdem, dass für eine Teilmenge
    $A'\subset\R^d$ die konvexe Hülle $\conv(A')$ gerade aus allen 
    Konvexkombinationen endlich vieler Elemente aus $A'$ besteht.
\end{thBemerkung}


\begin{thDef}[(Geometrisches) Simplex, Eckpunkte, Dimension]
    \label{gsc:def:simplex}
    %
    Sind $n,d\in\N$ und $v_0,\ldots,v_n\in\R^d$ affin unabhängig, so bezeichnet
    man $\sigma \defeq \conv(\{v_0,\ldots,v_n\})$ als \emph{(geometrisches)
    Simplex} und nennt die $v_i$ \emph{Eckpunkte (von $\sigma$)}, oder kurz
    \emph{Ecken (von $\sigma$)}. Die \emph{Dimension von $\sigma$} ist dann
    durch $\dim(\sigma) \defeq n$ definiert und man nennt ein $n$-dimensionales
    Simplex auch kurz \emph{$n$-Simplex}.
\end{thDef}

\begin{thDef}[Seite eines Simplex]
    Sind $n,d\in\N$ und ist $\sigma$ ein $n$-Simplex mit der Eckenmenge
    $E \subset\R^d$, so ist die konvexe Hülle $\conv(E')$ einer beliebigen 
    Teilmenge $E'\subset E$ wieder ein Simplex und jedes dieser Simplizes wird
    als \emph{Seite von $\sigma$} bezeichnet.
\end{thDef}

% TODO: Skizzen der 0 bis 3-dimensionalen Simplizes

\begin{thDef}%
    [Geometrischer simplizialer Komplex, Dimension, Polyeder, Eckenmenge]
    \label{gsc:def:gsc}
    %
    Sei $d\in\N$. Ist $\Delta$ eine nicht-leere Menge von (geometrischen)
    Simplizes im $\R^d$, so ist $\Delta$ ein \emph{(geometrischer) simplizialer
    Komplex}, falls folgende Bedingungen erfüllt sind:
    \begin{itemize}
        \item Ist $\sigma$ ein Simplex in $\Delta$, so ist auch jede Seite von
            $\sigma$ in $\Delta$ enthalten.
        \item
            Der Schnitt zweier Simplizes $\sigma_1,\sigma_2\in\Delta$ ist eine
            Seite von $\sigma_1$ und $\sigma_2$.
    \end{itemize}
    Wir definieren die \emph{Dimension von $\Delta$} als die maximale Dimension
    seiner Simplizies:
    \[ \dim(\Delta) \defeq \max\{ \dim(\sigma) \Mid \sigma\in\Delta \} \;\in\N
    \]
    Ist $\Delta$ ein simplizialer Komplex, so setzen wir
    \[ \polyeder\Delta \defeq \bigcup \Delta \;\subset\R^d \]
    und statten diesen Raum mit der von den Inklusionen der einzelnen Simplizes
    induzierten Finaltopologie aus. Explizit bedeutet das hier:
    $A\subset\polyeder\Delta$ ist genau dann offen (abgeschlossen), wenn für alle
    $\sigma\in\Delta$ der Schnitt $A\cap\sigma$ offen (abgeschlossen) in
    $\sigma$ ist (wobei $\sigma$ die Teilraumtopologie trägt). Wir nennen dann
    $\polyeder\Delta$ das \emph{zu $\Delta$ gehörige Polyeder}.
    Außerdem definieren wir $E(\Delta)$ als die Vereinigung
    aller Eckenmengen von Simplizes in $\Delta$.
\end{thDef}

\Achtung
In der Literatur wird des Öfteren der Fall des leeren Simplex~$\emptyset$
ausgeschlossen. Nach den obigen Definitionen \ref{gsc:def:simplex} und 
\ref{gsc:def:gsc} ist dies aber erlaubt und insbesondere enthält jeder

% TODO: Beispielskizzen

\begin{thDef}[Rand und Inneres eines Simplex]
    Ist $n\in\N$ und $\sigma$ ein $n$-Simplex, so bezeichnen wir die Vereinigung
    aller Seiten von $\sigma$ mit Dimension echt kleiner als $n$ als \emph{Rand
    von $\sigma$} und die Teilmenge von $\sigma$, die sich ergibt, wenn wir aus
    $\sigma$ den Rand entfernen, als \emph{(relatives) Inneres von $\sigma$}.
\end{thDef}

\begin{thDef}[Träger]
    Ist $\Delta$ ein simplizialer Komplex und $x\in\polyeder\Delta$, so gibt es
    genau ein Simplex $\sigma\in\Delta$, für dass $x$ im Inneren von $\sigma$
    liegt. Dieses bezeichnen wir mit $\supp(x)$.
\end{thDef}

\begin{thDef}[Teilkomplex, \texorpdfstring{$k$}{k}-Skelett]
    Sei $\Delta$ ein simplizialer Komplex. Wir nennen $\Delta'\subset\Delta$
    einen \emph{Teilkomplex}, falls $\Delta'$ selbst einen simplizialen Komplex
    definiert.
    Für jedes $k\in\setZeroto{\dim(\Delta)}$ ist
    \[ \skeleton\Delta{k} 
        \defeq \bigl\{ \sigma\in\Delta \Mid \dim(\sigma)\leq k \bigr\}
    \]
    ein Teilkomplex von $\Delta$, welchen wir \emph{$k$-Skelett von $\Delta$}
    nennen. (Insbesondere gilt für $k=0$: $\skeleton\Delta0 = E(\Delta)$.)
\end{thDef}

\bigskip
Im Folgenden werden wir hauptsächlich \emph{endliche} simpliziale Komplexe
betrachten, d.\,h. simpliziale Koplexe mit nur endlich vielen Simplizes. Daher
halten wir noch folgende Aussage fest:

\begin{thLemma}[Polyeder eines endlichen simplizialen Komplexes]
    Ist $d\in\N$ und $\Delta$ ein endlicher simplizialer Komplex im $\R^d$, 
    so stimmt die Topologie auf $\polyeder\Delta$ aus \cref{gsc:def:gsc} mit 
    der Teilraumtopologie von $\polyeder\Delta$ als Teilmenge des $\R^d$
    überein.
\end{thLemma}

Der Beweis ist einfach und wird an dieser Stelle dem Leser zur Übung überlassen.
Einfache aber nützliche Beispiele für endliche simpliziale Komplexe liefert das
folgende Lemma:

\begin{thLemma}[Simplizialer Komplex eines Simplex]
    Sind $n,d\in\N$ und ist $\sigma$ ein $n$-Simplex im $\R^d$, so ist
    \[ \Delta(\sigma) 
        \defeq \{ \sigma' \Mid \sigma' \text{ ist Seite von } \sigma \}
    \]
    ein $n$-dimensionaler (endlicher) simplizialer Komplex im $\R^d$.
\end{thLemma}

\begin{proofsketch}
    Seien $n,d,\sigma$ wie oben. Falls $\Delta(\sigma)$ ein simplizialer Komplex
    ist, ist klar, dass dieser endlich ist, da die Anzahl der Ecken von $\sigma$
    endlich ist. Außerdem gilt $\sigma\in\Delta(\sigma)$, womit
    $\dim\bigl(\Delta(\sigma)\bigr) = n$ klar ist. Es bleibt also zu zeigen,
    dass $\Delta(\sigma)$ einen simplizialen Komplex definiert. Die erste
    Bedingung in \cref{gsc:def:gsc} ist dabei offensichtlich erfüllt und für die
    zweite zeigt man mithilfe von \cref{gsc:convexhullviaconvexcombinations} 
    und einfachen Umformungen, dass für Teilmengen $A,B$ der
    Eckenmenge von $\sigma$ stets $\conv(A)\cap\conv(B) = \conv(A\cap B)$ gilt.
    \\
\end{proofsketch}

Das letzte Argument findet man ausführlicher auch bei 
Matou\v sek\cite[Lemma~1.3.6]{bookc:matousek03}.


\section{Triangulierung}
\begin{thDef}[Triangulierbar, Triangulierung]
    Ein beliebiger topologischer Raum~$X$ heißt \emph{triangulierbar}, falls es
    einen simplizialen Komplex $\Delta$ und einen Homöomorphismus
    $\polyeder\Delta \cong X$ gibt. In solch einem Fall nennen wir $\Delta$ eine
    \emph{Triangulierung von $X$}.
\end{thDef}

\Achtung
In der Literatur wird häufig nicht nur der simpliziale Komplex als
Triangulierung bezeichnet, sondern auch ein zugehöriger, fest gewählter
Homöomorphismus. Wir halten uns hier jedoch an die Bezeichnungen wie bei
Matou\v sek\cite{bookc:matousek03}.


